\chapter{Introduction}

The fascination with the idea of artificial intelligence and sentient robots dates back at least as far as ancient Greek mythology, which tells the story of Hephaestus, the Greek god of smithing, and his creation of Talos, a giant constructed of bronze to act as a guardian for the island of Crete.
This fascination could be observed throughout history, where the concept of thinking machines has captured the imagination of mathematicians, philosophers, and scientists. These brilliant minds pushed, throughout the years, the boundaries of what was possible, bringing artificial intelligence into the modern day, where it can drive a vehicle, write code, compose symphonies, and create astounding works of art.

Games and puzzles played an extremely significant part in this advancement, as the pioneers of the field recognized them as a great platform on which to conduct research. Alan Turing, for example, spent many years working on a chess program called Turochamp \cite{turing1953digital}. To date, Turochamp stands as the oldest known computer game to have begun development. However, its algorithm was too complex for the primitive computers of the day, and the program was never completed. Turing also considered other games, such as go and poker, but his research on these games was never published. Claude Shannon, known as the father of information theory, also worked on chess and published the first paper on the topic in 1949. Shannon opened his paper by stating how solving chess will pave the way to solving more complicated problems and enable the development of machines capable of translating different languages, making strategic decisions in military operations, and orchestrating melodies \cite{shannon1950chess}. Shannon was able to estimate the computational complexity of chess and understood that the game could not be solved using brute-force search approaches. In 1951 Christopher Strachey started working on a program that could play the game of checkers. By the summer of 1952, Strachey ran the earliest successful AI program capable of playing a complete game of checkers at a reasonable speed. 

These games provided a fertile environment for the growth of the field of artificial intelligence, as they have characteristics that make them suitable for acquiring experiences that can be transferred to more valuable domains of the real world. However, not all games were treated in the same way, as games such as chess and checkers received a lot of attention due to the ease of achieving higher performance through brute-force search \cite{billings1998poker}. Games such as poker, on the other hand, which are more complex and require more sophisticated approaches, had only attracted limited attention \cite{billings2006algorithms} until recently, when it was identified as a beneficial domain for AI research \cite{billings1998poker}. Traditional strategies like heuristic search and evaluation methods are inadequate to overcome the sheer complexity of the game \cite{billings_challenge_2002}. Moreover, the properties of the game, such as imperfect information, stochastic outcomes, risk management, deception, and unreliable information, make it a very difficult computational problem to solve. Solving poker will provide important insights into dealing with deliberate misinformation and making intelligent decisions despite incomplete information \cite{billings2006algorithms}. Several approaches have been used to solve poker, such as knowledge-based agents, enhanced simulation systems, theoretic equilibrium solutions, and exploitative counter-strategies \cite{rubin2011computer}.

More recently, the \gls{cfr} algorithm was utilized by bots such as DeepStack \cite{moravvcik2017deepstack} and Pluribus \cite{brown2019superhuman} to achieve significant breakthroughs, elevating AI agents to the level of professional poker players.

Different strategies and methods have been utilized in these approaches, but to the best of our knowledge, HTN planners have never been used. \gls{htn} are structures utilized by AI planning systems that offer a high level of expressiveness \cite{erol1994htn} and are widely implemented in real-world applications. It was previously \cite{alnazer2022risk} demonstrated that HTN planning can become risk-aware. Poker is a game centered around making decisions under uncertainty, and risk is an essential element in the game. HTN planning techniques provide a wide range of features, such as the capability of encoding expert knowledge, risk awareness and utility, and the ability to introduce abstraction levels, making it easy to break the problem into smaller segments. These features make HTN planning a potentially great framework for developing a risk-aware agent that can play poker competently. In this thesis, we will explore the use of HTN planning for playing Texas Hold'em as it is considered to be the most strategically demanding variation of poker, which will allow us to investigate the potential of this approach.


The remainder of this thesis is organized as follows. In the first chapter, we introduce the background information on the different topics related to this study, including poker, automated planning, and risk awareness. In the following chapter, we discuss the specifics of our implementation, including a review of HTN planners and the approaches we utilized to model the opponents. After that, we break down the playing strategies of the bot. Finally, in the last chapter, we provide an evaluation of the approach and a conclusion.