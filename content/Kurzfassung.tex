In der Geschichte der Computerwissenschaft waren Spiele immer ein wesentlicher Faktor bei der Weiterentwicklung des Bereichs der künstlichen Intelligenz. Die Pioniere der Informatik haben die besondere Wichtigkeit von Spielen wie Schach, Dame und anderen ähnlichen Spielen erkannt und jahrelang daran geforscht, sie zu verstehen. Diese Spiele boten einen hervorragenden Forschungsrahmen mit klar definierten Regeln, eindeutigen Zielen und einem geeigneten Messinstrument für die Leistung. Es wurde jedoch auch erkannt, dass perfekte Informationsspiele nicht den Entscheidungsprozess im wirklichen Leben widerspiegeln, wo Agenten Ungewissheit in Kauf nehmen und oft mit Fehlinformationen und Täuschung umgehen müssen. Das Pokerspiel modelliert erfolgreich diese Aspekte der Entscheidungsfindung und kann verwendet werden, um das Verhalten intelligenter Agenten in realen Domänen zu untersuchen. Poker hat in letzter Zeit die Aufmerksamkeit der Forscher zurückgewonnen, was zu bedeutenden Durchbrüchen geführt hat, die KI-Agenten auf das Niveau von Profispielern gehoben haben. Zur Lösung des Pokerspiels wurden verschiedene Ansätze verwendet, z. B. wissensbasierte Agenten, verbesserte Simulationssysteme, theoretische Gleichgewichtslösungen und exploitative Gegenstrategien. Unseres Wissens nach wurden jedoch noch nie HTN-Planungstechniken eingesetzt. In dieser Arbeit wird der Einsatz von risikobewusster HTN-Planung für no limit Texas Hold'em, der strategisch anspruchsvollsten Pokervariante, untersucht, um das Potenzial dieses Ansatzes zu erforschen.