\subsection{Classical Planning}

As a means of introduction, we will explore classical planning, 
which refers to the most basic form of automated planning.
In order to avoid the complexity of the real world, classical 
planning deals with an abstract model of the environment, 
imposing a number of restrictive assumptions:

\begin{itemize}
    \item 
    \textbf{Finite, observable, and static:}
    The environment has only a finite number of states and actions. The current state of the environment is known to the agent, and it can not be changed without an action initiated by the agent.

    \item 
    \textbf{Deterministic, no uncertainty:}
    The agent can predict with certainty the changes in the current state if a given action is performed.

    \item 
    \textbf{Implicit time:}
    No explicit model of time. There is only a linear sequence of instantaneous states.
\end{itemize}

We can now introduce the formal definition of classical planning. 
The formal structures we use are adapted from formal definitions 
described in \cite{AutomatedPlanningTheoryghallab2006}\cite{OverviewHierarchicalTaskgeorgievski2014}\cite{alnazer2019htn}:


\begin{Tdef}[Predicate]
    A predicate $p$ is defined as follows: $p= \langle symbol(p), terms(p) \rangle$, where:
    \vspace{-0.5em}
    \begin{compactitem} 
        \item 
        $symbol(p)$ is the predicate symbol.
        \item 
        $terms(p)$ is a set of terms such that $terms(p) = \langle \tau_1, \tau_2, \dots, \tau_n \rangle$.
        \item 
        A term is either:
        \begin{compactitem}
            \item 
            A constant $co \in \mathcal{CO}$, where $\mathcal{CO}$ is a finite set of constants.
            \item 
            A variable $v \in \mathcal{V}$, where $\mathcal{V}$ is an infinite set of variables.
        \end{compactitem}
    \end{compactitem}
    \vspace{-0.5em}
    Predicates can be used to describe the objects of the environment and their relations.
\end{Tdef}

\begin{Tdef}[Ground Predicate]
    A ground predicate is a predicate where all the terms are constants.
\end{Tdef}
    
\begin{Tdef}[State]
    A state $s$ is defined as a set of ground predicates: $s = \left\{ p_1, p_2, \dots, p_n\right\}$
    while describing a state, we adopt the \textit{closed world assumption}\footnote{The state description only includes predicates that evaluate to true}.
\end{Tdef}
    
\begin{Tdef}[Action]
    An Action $a$ is defined as follows: $a = \langle pre(a), eff(a) \rangle$, where:
    \vspace{-0.5em}
    \begin{compactitem}
    \item 
    $pre(a)$ is the precondition that has to hold to perform the action.
    \item 
    $eff(a)$ is the effect of the action on the state.
    \end{compactitem}
\end{Tdef}

\begin{Tdef}[Applicable Action]
    Given a state $s$ and an action $a$, we say that $a$ is applicable in $s$ 
    if and only if:
    \vspace{-0.5em}
    $$\forall p_i \in pre(a): p_i\in s \text{ and } \forall \bar{p_i} \notin s$$
    Applying action $a$ in state $s$ will result in a new state $\acute{s}$
    in which some predicates are deleted 
    (i.e., are given the boolean value false) while others are added.
    The set of deleted and added predicates are denoted by $eff^-(a), eff^+(a)$, respectively.
    We denote the application of an action $a$ to a state $s$ as:
    \vspace{-0.5em}
    $$\gamma(s,a) := (s \cup eff(a)^+) \setminus eff(a)^- = \acute{s}$$
    For a sequence of actions $\mathbf{A} = \langle a_1,\dots,a_n \rangle$ we recursively define 
    the application of $\mathbf{A}$ to $s$ as :
    \vspace{-0.5em}
    \begin{compactitem}
    \item 
    $\hat{\gamma}(s,\langle \rangle) := s$
    \item 
    $\hat{\gamma}(s,\langle a_1,\dots,a_n\rangle) := \hat{\gamma}(\gamma(s,a_1),\langle a_2,\dots,a_n\rangle)$
    \end{compactitem}
    given that $\gamma$ is well defined for each $a_i$.
\end{Tdef}



\subsubsection{Planning domain}

For an intelligent agent to be able to conduct a reasoning process, 
it should have a model of the environment. 
This model does not have to reflect each element of the environment, 
but it should be a good and accurate approximation.
\begin{Tdef}[Planning Domain]

    A planning domain is a tuple $\Sigma= (\mathcal{P}, \mathcal{A})$ where:
    \vspace{-0.5em}
    \begin{compactitem}
        \item 
        $\mathcal{P}$ is a finite set of predicates.
        \item 
        $\mathcal{A}$ is a finite set of actions.
        \item 
        The finite set $\mathcal{S}$, which contains all possible states, can be inferred from $\mathcal{P}$ and $\mathcal{A}$.
    \end{compactitem}
    \vspace{-0.5em}
\end{Tdef}

In practice, \gls{pddl} are used to describe the domain \cite{aeronautiques1998pddl}. There are many different versions and variations of \gls{pddl}, each with a certain set of capabilities and limitations. 
Listing~\ref{lst:PDDLDomainDefinitonExample} shows a simple example from \cite{IntroductionPlanningDomainhaslum2019}, where a description of a simple switch is provided with two actions:

\begin{Listing}
    \begin{lstlisting}[language=PDDL]
    (define (domain switch)
        (:requirements :strips)
        (:predicates (switch_is_on)(switch_is_off))
        (:action switch_on
            :precondition (switch_is_off)
            :effect (and (switch_is_on) (not (switch_is_off))))
        (:action switch_off
            :precondition (switch_is_on)
            :effect (and (switch_is_off) (not (switch_is_on)))))
  \end{lstlisting}
    \caption{PDDL domain definition example adopted from \cite{IntroductionPlanningDomainhaslum2019}.}
    \label{lst:PDDLDomainDefinitonExample}
\end{Listing}

\subsubsection{Planning problem}
A planning domain is not a description of the current state of the environment 
but rather a model of all the possible states of the environment. 
Nevertheless, this is insufficient for the agent. The agent must be aware 
of the current state of the environment and have a clear definition 
of what to achieve.

\begin{Tdef}[Planning Problem]
    A planning problem is a triple $\pi = (\Sigma,s_I,s_G)$, where :
    \vspace{-0.5em}
    \begin{compactitem}
        \item 
        $\Sigma$ is a problem domain.
        
        \item 
        $s_I$ is the initial state.
        
        \item 
        $s_G$ is the goal state.
    \end{compactitem}
    \vspace{-0.5em}
\end{Tdef}
\vspace{-0.5em}
Just like planning domains, planning problems are described in practice using \gls{pddl}. 
Listing~\ref{lst:PDDLProblemDefinitonExample} shows a problem definition for the previously defined domain:
\begin{Listing}
    \begin{lstlisting}[language=PDDL]
    (define (problem turn_it_off)
        (:domain switch)
        (:init (switch_is_on))
        (:goal(switch_is_off)))
  \end{lstlisting}
    \caption{PDDL problem definition example}
    \label{lst:PDDLProblemDefinitonExample}
\end{Listing}

Appendix~\ref{appendix:blocksWorldDomain} contains a more complex example of a planning domain and problem.

\begin{Tdef}[Valid Plan]
    Given a planning domain $\Sigma$ and a planning problem $\pi$,
    a plan $\mathbf{P} = \langle a_1,a_2,\dots,a_n\rangle$ is a valid plan for $\pi$ if:
    $$\hat{\gamma}(s_I,\mathbf{P}) = s_n \in s_G$$
\end{Tdef}

\subsubsection{Planners}

A planner takes a planning problem and formulates a sequence of actions 
to transform the current state of the environment into a state 
that satisfies the objective. Various planners employ different problem-solving strategies. 
For example, some planners use theorem proving \cite{PlanningSatisfiabilitykautz1992} 
to generate a plan, whereas others use state-space search. 
Planning is computationally hard 
\cite{ComplexityDecidabilityUndecidabilityerol1995}, and 
sometimes it is not enough to find a plan that achieves the goal, 
but the planner has to consider also a set of metrics that make plan 
\textit{a} better/worse than plan \textit{b}, which causes the planning process to be more and more complicated.
